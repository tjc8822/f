\documentclass[9pt]{beamer}
\usepackage{soul}
\newcommand{\mathcolorbox}[2]{\colorbox{#1}{$\displaystyle #2$}}
\usetheme{simple}
\usepackage{lmodern}
\usepackage{booktabs}
\usepackage[scale=2]{ccicons}
\newcommand{\hc}{\hat{c}}
\newcommand{\hpi}{\hat{\pi}}
\newcommand{\hbeta}{\hat{\beta}}
\newcommand{\hphi}{\hat{\phi}}
\newcommand{\hU}{\hat{U}}
\newcommand{\ha}{\hat{a}}
\newcommand{\hsigma}{\hat{\sigma}}
\newcommand{\hv}{\hat{v}}
\newcommand{\hR}{\hat{R}}
\newcommand{\hK}{\hat{K}}
\newcommand{\hM}{\hat{M}}
\newcommand{\yt}{y^t}
\newcommand{\ytp}{y^{t+1}}
\newcommand{\zt}{z^t}
\newcommand{\ztp}{z^{t+1}}
\newcommand{\bewley}{\textit{Bewley model$\hspace{1.5mm}$}}
\newcommand{\arrow}{\textit{Arrow model$\hspace{1.5mm}$}}
\newcommand{\bond}{\textit{Bond model$\hspace{1.5mm}$}}
\usepackage{amsmath}
\definecolor{destacado}{HTML}{70291d} % gris oscuro
  \definecolor{normal}{HTML}{797979}  % gris medio
  \definecolor{fondo}{HTML}{e6e6e6}  % gris claro
  \definecolor{background}{HTML}{fbfaf6}  % gris claro
\makeatletter
\def\th@mystyle{%
    \normalfont % body font
    \setbeamercolor{block title example}{bg=orange,fg=white}
    \setbeamercolor{block body example}{bg=orange!20,fg=black}
    \def\inserttheoremblockenv{exampleblock}
  }
\makeatother
\theoremstyle{mystyle}
\newtheorem*{remark}{Remark}


% TODO: 
%   position adjustement
%   change colours
%       

% Watermark background (simple theme)



\title{When is market incompleteness irrelevant for the price
of aggregate risk (and when is it not)?}
\subtitle{Krueger $\&$ Lustig 2010 JET}
\date{\today}
\author{Presented by Jincheng Tong}

\begin{document}
\maketitle
%%%%%%%%%%%%%%%%%%%%%%%%%%%%%%%%%%%%%%%%%%%%%%%%%%%%%%%%%%%%%%%%%%%%%%%%%%%%%%%%%%%%%%%%%%%%%%%%%%
\begin{frame}{What is this paper about}
\begin{itemize}
\item for a large class of incomplete market models with aggregate and idiosyncratic risks,  characterize the conditions under which 
\vspace{5mm}
\begin{itemize}
\item equilibrium prices and quantities can be obtained from those in a \bewley with only idiosyncratic risk 
\vspace{5mm}
\item make contact to the aggregation literature
\end{itemize}
\vspace{5mm}
\item given the equivalence results, heterogeneity does not matter for 
\vspace{5mm}
\begin{itemize}
\item equity premium
\vspace{5mm}
\item wealth inequality over business cycles
\end{itemize}
\end{itemize}
\end{frame}
%%%%%%%%%%%%%%%%%%%%%%%%%%%%%%%%%%%%%%%%%%%%%%%%%%%%%%%%%%%%%%%%%%%%%%%%%%%%%%%%%%%%%%%%%%%%%%%%%%
\begin{frame}{Outline}
\begin{enumerate}
\item quickly review some notations
\vspace{5mm}
\item the general results in the infinite horizon economy:
\vspace{5mm}
\begin{itemize}
\item discuss the equilibrium concept carefully in the \bewley
\vspace{5mm}
\item show how to construct equilibrium allocations and prices in the \arrow using the equilibrium outcomes in the \bewley 
\end{itemize}
\vspace{5mm}
\item if we have time, bridge the gap between the two-period example and the general results

\end{enumerate}
\end{frame}
%%%%%%%%%%%%%%%%%%%%%%%%%%%%%%%%%%%%%%%%%%%%%%%%%%%%%%%%%%%%%%%%%%%%%%%%%%%%%%%%%%%%%%%%%%%%%%%%%%
\begin{frame}{Environment}
\begin{itemize}%[<+->]
\item aggregate shock $z_t$, 
\vspace{5mm}
\item idiosyncratic shock $y_t$
\vspace{5mm}
\item  $s_t = (y_t,z_t),  s^t = (y^t,z^t)$
\vspace{5mm}
\item aggregate endowment $e_t(z^t)$, its growth rate $\lambda(z_{t+1}) = \frac{e_{t+1}(z^{t+1})}{e_{t}(z^{t})}$

\vspace{5mm}
\item $\alpha(z_t)e_t(z^t)$ is capital income: stock represents a claim to capital income
\vspace{5mm}
\item $(1-\alpha(z_t))e_t(z^t)$ is labor income: individual labor income process $\{\eta_t\}$
\begin{equation*}
\eta_t(s^t) = (1-\alpha(z_t))e_t(z^t) \times \eta(y_t, z_t)
\end{equation*}
\end{itemize}
\end{frame}

%%%%%%%%%%%%%%%%%%%%%%%%%%%%%%%%%%%%%%%%%%%%%%%%%%%%%%%%%%%%%%%%%%%%%%%%%%%%%%%%%%%%%%%%%%%%%%%%%%

\begin{frame}{infinite horizon: key assumptions}
\begin{itemize}
\item independence of idiosyncratic shocks from aggregate conditions:\vspace{5mm} 
\begin{itemize}
\item individual labor endowment shares:  $\eta(y_t, z_t) = \eta(y_t)$
\vspace{5mm}
\item for the growth-adjusted probability 

\begin{equation*}
\hpi(s_{t+1}|s_t)=\psi(y_{t+1}|y_t)\hphi(z_{t+1}|z_t), \quad\text{and}\quad \hphi(z_{t+1}|z_t) = \frac{\phi(z_{t+1}|z_t)\lambda(z_{t+1})^{1-\gamma}}{\sum_{z_{t+1}}\phi(z_{t+1}|z_t)\lambda(z_{t+1})^{1-\gamma}}
\end{equation*}
\end{itemize}
\vspace{2mm}
\item aggregate endowment growth is i.i.d: $\phi(z_{t+1}|z_t)=\phi(z_{t+1})$
\vspace{5mm} 
\begin{itemize}
\item $\hphi(z_{t+1}|z_t)=\hphi(z_{t+1})$, $\hbeta(z_t) = \hbeta$
\end{itemize}
\vspace{5mm}
\item the labor income share $1-\alpha$ is constant
\end{itemize}
\end{frame}
%%%%%%%%%%%%%%%%%%%%%%%%%%%%%%%%%%%%%%%%%%%%%%%%%%%%%%%%%%%%%%%%%%%%%%%%%%%%%%%%%%%%%%%%%%%%%%%%%%
\begin{frame}{\textit{the Bewley model}}
\begin{itemize}
\item no aggregate uncertainty; agents can trade stock and bond
\vspace{2mm}
\item agents maximize utility given budge constraints
\begin{equation*}
\hc_t(y^t) + \frac{\ha_t(y^t)}{\hR_t} + \hsigma_t(y^t)\hv_t = (1-\alpha)\eta(y_t) + \ha_{t-1}(y^{t-1})+\hsigma_{t-1}(y^{t-1})(\hv_t+\alpha)
\end{equation*}
\vspace{2mm}
\item agents also face one of two types of borrowing constraints (or both)
\begin{equation*}
\begin{split}
&\frac{\ha_t(\yt)}{\hR_t}+\hsigma_t(\yt)\hv_t\geq \hK_t(\yt)\quad \forall \yt \\
&\ha_t(\yt) + \hsigma_t(\yt)(\hv_{t+1}+\alpha)\geq \hM_t(\yt)\quad \forall \yt
\end{split}
\end{equation*}
\vspace{2mm}
\item the two solvency constraints are identical in the \bewley but it will be useful in models with aggregate uncertainty 
\end{itemize}
\end{frame}
%%%%%%%%%%%%%%%%%%%%%%%%%%%%%%%%%%%%%%%%%%%%%%%%%%%%%%%%%%%%%%%%%%%%%%%%%%%%%%%%%%%%%%%%%%%%%
\begin{frame}{Equilibrium in \bewley}
for an initial distribution over $(\theta_0, y_0)$, a \textbf{competitive equilibrium} for the \bewley consists of 
\begin{itemize}
\vspace{3mm}
\item allocations: $\{ \hc_t(\theta_0,y^t), \ha_t(\theta_0,y^t),\hsigma_t(\theta_0,y^t)  \}$; prices $\{ \hv_t,\hR_t\}$
\end{itemize}
\vspace{3mm}
such that 
\vspace{3mm}
\begin{itemize}
\item given prices, allocations solve the household maximization problem
\vspace{5mm}
\item the markets clear in \textbf{all periods} $t$
\vspace{3mm}
\begin{itemize}
\item the goods markets: $\int \sum_{y^t}\phi(y^t|y_0)\hc_t(\theta_0,y^t)d\Theta_0=1$
\vspace{3mm}
\item the bonds markets: $\int \sum_{y^t}\phi(y^t|y_0)\ha_t(\theta_0,y^t)d\Theta_0=0$
\vspace{3mm}
\item the stock markets: $\int \sum_{y^t}\phi(y^t|y_0)\hsigma_t(\theta_0,y^t)d\Theta_0=1$
\end{itemize}
\end{itemize}
\vspace{3mm}
\begin{itemize}
\item a \textit{stationary equilibrium} in the \bewley consists of $\{\hR, \hv\}$, optimal household allocations and a time-invariant measure $\Phi$ over income shocks and wealth
\vspace{3mm}
\item normalize the bond positions to be 0
\end{itemize}
 \end{frame}
%%%%%%%%%%%%%%%%%%%%%%%%%%%%%%%%%%%%%%%%%%%%%%%%%%%%%%%%%%%%%%%%%%%%%%%%%%%%%%%%%%%%%%%%%%%%%
\begin{frame}{\arrow} 
\begin{itemize}
\item with aggregate uncertainty: agents can trade shares of stocks and contingent claims on aggregate states
\vspace{5mm}
\item agents maximize utility given budget constraints
\begin{equation*}
c_t(s^t) + \sum_{z_{t+1}}a_t(s^t,z_{t+1})q_t(z^t,z_{t+1}) + \sigma_t(s^t)v_t(z^t)\leq\theta_t(s^t) 
\end{equation*}
\begin{equation*}
\begin{split}
 \theta_{t+1}(s^{t+1}) = (1-\alpha)&\eta(y_{t+1},z_{t+1})e_{t+1}z_{t+1} \\&+ a_t(s^t,z_{t+1}) + \sigma_t(s^t)[v_{t+1}(z^{t+1})+\alpha e_{t+1}(z_{t+1})]
\end{split}
\end{equation*}
\vspace{3mm}
\item solvency constraints: agents also face one of two types of borrowing constraints (or both)
\begin{equation*}
\begin{split}
\sum_{z_{t+1}}a_t(s^t,z_{t+1})q_t(\zt,z_{t+1})+\sigma_t(s^t)v_t(\zt)\geq \hK_t(\yt)e_t(z^t)
\end{split}
\end{equation*}
\begin{equation*}
\begin{split}
a_t(s^t,z_{t+1})+\sigma_t(s^{t})[v_{t+1}(\ztp)+\alpha e_{t+1}(z_{t+1})]\geq \hM_t(\yt)e_{t+1}(z^{t+1})
\end{split}
\end{equation*}
\vspace{5mm}
\item can de-trend the \arrow by the aggregate endowment
\end{itemize}
\end{frame}
%%%%%%%%%%%%%%%%%%%%%%%%%%%%%%%%%%%%%%%%%%%%%%%%%%%%%%%%%%%%%%%%%%%%%%%%%%%%%%%%%%%%%%%%%%%%%
\begin{frame}{equivalence results}
\begin{theorem}
If $\{ \hc_t(\theta_0,y^t), \ha_t(\theta_0,y^t), \hsigma_t(\theta_0,y^t)\}$ and $\{\hR_t, \hv_t\}$ is an equilibrium of the \bewley, then$ \{a_t(\theta_0,s^t,z_{t+1})), \sigma_t(\theta_0,s^t), c_t(\theta_0, s^t)\}$ and $\{q_t(z^t,z_{t+1}), v_t(z^t) \}$ is an equilibrium of the stochastically growing \arrow with
\vspace{3mm}
\begin{itemize}
\item $c_t(\theta_0,s^t) = \hc(\theta_0,y^t)e_t(z^t)$
\vspace{3mm}
\item $\sigma_t(\theta_0,s^t)=\hsigma_t(\theta_0,y^t)$
\vspace{3mm}
\item $a_t(\theta_0, s^t, z_{t+1}) = \ha_t(\theta_0, y^t)e_{t+1}(z^{t+1})$
\vspace{3mm}
\item $v_t(z^t)=\hv_t e_t(z^t)$
\vspace{3mm}
\item $q_t(z^t, z_{t+1})= \frac{1}{\hR_t}\times \frac{\phi(z_{t+1})\lambda(z_{t+1})^{-\gamma}}{\sum_{z_{t+1}}\phi(z_{t+1})\lambda(z_{t+1})^{1-\gamma}}$
\end{itemize}
\end{theorem}
\vspace{5mm}
\begin{itemize}
\item in a \textbf{de-trended} \arrow, its equilibrium allocations are identical to those in a  \bewley 
\vspace{5mm}
\item aggregate uncertainty does not interact with normalized wealth distribution
\end{itemize}
\end{frame}
%%%%%%%%%%%%%%%%%%%%%%%%%%%%%%%%%%%%%%%%%%%%%%%%%%%%%%%%%%%%%%%%%%%%%%%%%%%%%%%%%%%%%%%%%%%%%
\begin{frame}{sketch of proof}
\begin{itemize}
\item to jointly solve for allocations and prices is complicated in \arrow
\vspace{5mm}
\begin{itemize}
\item conjecture a constant pd ratio in \arrow
\vspace{5mm}
\item guess that Arrow securities' prices in \arrow is given by multiplying Arrow securities' prices in a representative agent economy with some constant
\vspace{5mm}
\item plug the conjectured prices into the Euler equations and complementary-slackness conditions in the \arrow
\vspace{5mm}
\item it turns out that the optimal allocations in the \bewley satisfies those optimality conditions, under conjectured prices
\vspace{5mm}
\item also market clearing conditions coincide in the \bewley and de-trended \arrow
\end{itemize}
\end{itemize}
\end{frame}
%%%%%%%%%%%%%%%%%%%%%%%%%%%%%%%%%%%%%%%%%%%%%%%%%%%%%%%%%%%%%%%%%%%%%%%%%%%%%%%%%%%%%%%%%%
\begin{frame}{intuition for the i.i.d aggregate endowment growth assumption}
\begin{itemize}
\item the i.i.d aggregate endowment growth assumption alone may already imply that the distribution is not a relevant state variable
\vspace{5mm}
\begin{itemize}
\item with i.i.d aggregate endowment growth rate, no predictable component in agents' wealth;
\vspace{5mm}
\item therefore, there is no predictable component in individual agents' marginal utility (value) of wealth; 
\vspace{5mm}
\item without such predicable component in wealth due to the lack of predictability in aggregate shock, we expect that once scaled out aggregate endowment, the normalized wealth distribution coincide with that in a \bewley. Therefore, the wealth distribution does not matter.
\vspace{5mm}
\end{itemize}
\end{itemize}
\end{frame}
%%%%%%%%%%%%%%%%%%%%%%%%%%%%%%%%%%%%%%%%%%%%%%%%%%%%%%%%%%%%%%%%%%%%%%%%%%%%%%%%%%%%%%%%%%
\begin{frame}{the connection between the general results to the two period model in the paper}

\begin{itemize}
\item recall that for the two period model: ex-ante homogeneity leads to a no trade (in stocks and bonds) autarkic equilibrium
\vspace{5mm}
\item from the general results to the allocations in the two period model
\vspace{5mm}
\begin{itemize}
\item start with the same degenerate wealth distribution as in the 2-period model
\vspace{5mm}
\item compute the competitive equilibrium at $t=0$ in the \bewley
\vspace{5mm}
\item scale people's allocations and prices properly by the main theorem
\end{itemize}
\end{itemize}

\end{frame}

%%%%%%%%%%%%%%%%%%%%%%%%%%%%%%%%%%%%%%%%%%%%%%%%%%%%%%%%%%%%%%%%%%%%%%%%%%%%%%%%%%%%%%%%%%
\begin{frame}{infinite horizon: notations }
\begin{itemize}
\item transformation of the growth model into a stationary model as in Alvarez Jermann (2001)
\begin{equation*}
\hc(s^t) = \frac{c_t(s^t)}{e_t(z^t)}
\end{equation*}
\begin{itemize}
\item growth-adjusted probabilities and discount factor
\vspace{2mm}
\begin{equation*}
\hpi(s_{t+1}|s_t) = \frac{\pi(s_{t+1}|s_t)\lambda(z_{t+1})^{1-\gamma}}{\sum_{s_{t+1}}\pi(s_{t+1}|s_t)\lambda(z_{t+1})^{1-\gamma}}\quad \hbeta(s_t) = \beta\sum_{s_{t+1}}\pi(s_{t+1}|s_t)\lambda(z_{t+1})^{1-\gamma}
\end{equation*}
\item life-time utility
\begin{equation*}
\hU(\hc)(s^t) = u(\hc(s^t)) + \hbeta(s_t) \sum_{s_{t+1}}\hpi(s_{t+1}|s_t)\hU(\hc)(s^t,s_{t+1})
\end{equation*}
\end{itemize}
\end{itemize}
\end{frame}
%%%%%%%%%%%%%%%%%%%%%%%%%%%%%%%%%%%%%%%%%%%%%%%%%%%%%%%%%%%%%%%%%%%%%%%%%%%%%%%%%%%%%%%%%%%%%%%%%%
%\begin{frame}
for a large class of incomplete market model with aggregate/idiosyncratic uncertainty, they look for a set of conditions on the structure of shocks, preference etc such so that the equilibirum prices and quantities can be constructed from those in a stationary model with only idiosyncratic uncertainty. The idea is to obtain the distribution of individual's wealth and shocks, then scale individual's consumption, asset by the aggregate endowment. It also means that the normalized wealth distribution does not interact with aggregate shock. 

obviously, the wealth distribution does not matter for market price of risk; risk premium is the same as the representative agent economy
\end{frame}
\begin{frame}
just introduce the outline
\end{frame}
\begin{frame}
introduce the environment
\end{frame}
\begin{frame}
introduce the two-period example
\end{frame}
\begin{frame}{no-trade equilibrium and the SDF}
Since there is no heterogeneity at time 0, different individuals are just replica of one person. So there is no trade in equity and Arrow securities market. About the aggregation, you can think of this example as that there is a representative agent who consumes aggregate endowment, but his labor income is subject to a random shock and such a shock is un-insurable. 

The ratio of Arrow securities price/ratio of IMRS for every agent here is the following, where $m$ is the IMRS the first argument is consumption at time 1 the second is consumption at time 0. 

For the representative agent model without un-insurable labor income risk, the ratio of

Let's compare 1 and 2 and figure out the set of conditions that make 1 = 2?
\end{frame}
%%%%%%%%%%%%%%%%%%%%%%%%%%%%%%%%%%
\begin{frame}{Irrelevance results}
independence of idiosyncratic shocks and aggregate shocks: meaning individual labor endowment share does not depend on aggregate state and also the distribution does not depend on aggregate shocks. However, this does not say that idiosyncratic shock is i.i.d; this rule out the counter-cyclical xs variance of income shocks mechanism as in CD

secondly, the capital income share cannot depend on aggregate state: you don't want the variation of individual consumption all comes from the variation of aggregate endowment so that later you can scale everyone's consumption or wealth by aggregate endowment

finally, because of the scaling results, you need homogeneity on marginal utility. Here we take CRRA utility for example 
\end{frame}
%%%%%%%%%%%%%%%%%%%%%%%%%%%%%%%%%%
\begin{frame}{robustness}
for incomplete market models, we expect some sort of borrowing constraints/solvency constraints: they add a set of constraint that says the bond plus stock value cannot be below some limits that is growing with aggregate endowment. This is a set of constraints that do not bind in equilibrium because of the no-trade results

there is a special case that can give you counter-cyclical labor income share: your individual labor income has two components: basic salary out-of aggregate labor share; the second component is like a bonus that transfers part of capital income, $\alpha-\alpha(z_1)$ to labor income; $\alpha(z_1)$ is the capital income share. 

\end{frame}
%%%%%%%%%%%%%%%%%%%%%%%%%%%%%%%%%
\begin{frame}{what is missing in the 2-period example?}
no trade equilibrium: ex-ante identical agents give rise to a no-trade equilibrium. But this may not be true in the infinite horizon; secondly, no variation/dispersion in wealth/consumption distribution over time; for non i.i.d shocks, 

\end{frame}
%%%%%%%%%%%%%%%%%%%%%%%%%%%%%%%%%
\begin{frame}
some notation: endowment is growing stochastically; consumption, asset position, wealth with a hat indicate that those are allocations that has been scaled by aggregate endowment; also, we need to adjust for the growth rate of the economy and the discount factor. They do this adjustment because they do not want to carry the growth rate term of around in the life-time utility function
\end{frame}
%%%%%%%%%%%%%%%%%%%%%%%%%%%%%%%%%
\begin{frame}
independence of idiosyncratic shocks and aggregate shocks: individual labor income share does not depend on aggregate states; in this case, the growth-adjusted probability matrix and the re-scaled discount factor is obtained by adjusting only the transition probabilities for the aggregate shock $\phi$, but not the transition probabilities for the idiosyncratic shocks.

aggregate endowment is i.i.d, so that the adjusted discount factor is constant
\end{frame}
%%%%%%%%%%%%%%%%%%%%%%%%%%%%%%%%%
\begin{frame}
just introduce the Bewley model
\end{frame}
%%%%%%%%%%%%%%%%%%%%%%%%%%%%%%%%%
\begin{frame}
define the equilibrium
\end{frame}
%%%%%%%%%%%%%%%%%%%%%%%%%%%%%%%%%
\begin{frame}
Introduce the Arrow mode
\end{frame}
%%%%%%%%%%%%%%%%%%%%%%%%%%%%%%%%%
\begin{frame}{equivalence results}
price dividend ratio is constant; because risk-free rate is constant in \bewley and i.i.d assumption, then the numerator is a constant: the Arrow securities is some constant multiply by the Arrow securities in the representative agent economy; 

the way to prove the theorem is that to jointly solve for allocations and prices is complicated; they guess that the pd ratio is constant. It is a reasonable guess because of i.i.d, constant capital share etc. They also guess the state-price densities is some scaled price densities as in the RA economy. Then they check that \bewley and \arrow, agents face the same set of Euler equations and complementarity slackness conditions
\end{frame}
%%%%%%%%%%%%%%%%%%%%%%%%%%%%%%%%%
\begin{frame}{equivalence results}

\end{frame}



\end{document}