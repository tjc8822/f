\begin{frame}
for a large class of incomplete market model with aggregate/idiosyncratic uncertainty, they look for a set of conditions on the structure of shocks, preference etc such so that the equilibirum prices and quantities can be constructed from those in a stationary model with only idiosyncratic uncertainty. The idea is to obtain the distribution of individual's wealth and shocks, then scale individual's consumption, asset by the aggregate endowment. It also means that the normalized wealth distribution does not interact with aggregate shock. 

obviously, the wealth distribution does not matter for market price of risk; risk premium is the same as the representative agent economy
\end{frame}
\begin{frame}
just introduce the outline
\end{frame}
\begin{frame}
introduce the environment
\end{frame}
\begin{frame}
introduce the two-period example
\end{frame}
\begin{frame}{no-trade equilibrium and the SDF}
Since there is no heterogeneity at time 0, different individuals are just replica of one person. So there is no trade in equity and Arrow securities market. About the aggregation, you can think of this example as that there is a representative agent who consumes aggregate endowment, but his labor income is subject to a random shock and such a shock is un-insurable. 

The ratio of Arrow securities price/ratio of IMRS for every agent here is the following, where $m$ is the IMRS the first argument is consumption at time 1 the second is consumption at time 0. 

For the representative agent model without un-insurable labor income risk, the ratio of

Let's compare 1 and 2 and figure out the set of conditions that make 1 = 2?
\end{frame}
%%%%%%%%%%%%%%%%%%%%%%%%%%%%%%%%%%
\begin{frame}{Irrelevance results}
independence of idiosyncratic shocks and aggregate shocks: meaning individual labor endowment share does not depend on aggregate state and also the distribution does not depend on aggregate shocks. However, this does not say that idiosyncratic shock is i.i.d; this rule out the counter-cyclical xs variance of income shocks mechanism as in CD

secondly, the capital income share cannot depend on aggregate state: you don't want the variation of individual consumption all comes from the variation of aggregate endowment so that later you can scale everyone's consumption or wealth by aggregate endowment

finally, because of the scaling results, you need homogeneity on marginal utility. Here we take CRRA utility for example 
\end{frame}
%%%%%%%%%%%%%%%%%%%%%%%%%%%%%%%%%%
\begin{frame}{robustness}
for incomplete market models, we expect some sort of borrowing constraints/solvency constraints: they add a set of constraint that says the bond plus stock value cannot be below some limits that is growing with aggregate endowment. This is a set of constraints that do not bind in equilibrium because of the no-trade results

there is a special case that can give you counter-cyclical labor income share: your individual labor income has two components: basic salary out-of aggregate labor share; the second component is like a bonus that transfers part of capital income, $\alpha-\alpha(z_1)$ to labor income; $\alpha(z_1)$ is the capital income share. 

\end{frame}
%%%%%%%%%%%%%%%%%%%%%%%%%%%%%%%%%
\begin{frame}{what is missing in the 2-period example?}
no trade equilibrium: ex-ante identical agents give rise to a no-trade equilibrium. But this may not be true in the infinite horizon; secondly, no variation/dispersion in wealth/consumption distribution over time; for non i.i.d shocks, 

\end{frame}
%%%%%%%%%%%%%%%%%%%%%%%%%%%%%%%%%
\begin{frame}
some notation: endowment is growing stochastically; consumption, asset position, wealth with a hat indicate that those are allocations that has been scaled by aggregate endowment; also, we need to adjust for the growth rate of the economy and the discount factor. They do this adjustment because they do not want to carry the growth rate term of around in the life-time utility function
\end{frame}
%%%%%%%%%%%%%%%%%%%%%%%%%%%%%%%%%
\begin{frame}
independence of idiosyncratic shocks and aggregate shocks: individual labor income share does not depend on aggregate states; in this case, the growth-adjusted probability matrix and the re-scaled discount factor is obtained by adjusting only the transition probabilities for the aggregate shock $\phi$, but not the transition probabilities for the idiosyncratic shocks.

aggregate endowment is i.i.d, so that the adjusted discount factor is constant
\end{frame}
%%%%%%%%%%%%%%%%%%%%%%%%%%%%%%%%%
\begin{frame}
just introduce the Bewley model
\end{frame}
%%%%%%%%%%%%%%%%%%%%%%%%%%%%%%%%%
\begin{frame}
define the equilibrium
\end{frame}
%%%%%%%%%%%%%%%%%%%%%%%%%%%%%%%%%
\begin{frame}
Introduce the Arrow mode
\end{frame}
%%%%%%%%%%%%%%%%%%%%%%%%%%%%%%%%%
\begin{frame}{equivalence results}
price dividend ratio is constant; because risk-free rate is constant in \bewley and i.i.d assumption, then the numerator is a constant: the Arrow securities is some constant multiply by the Arrow securities in the representative agent economy; 

the way to prove the theorem is that to jointly solve for allocations and prices is complicated; they guess that the pd ratio is constant. It is a reasonable guess because of i.i.d, constant capital share etc. They also guess the state-price densities is some scaled price densities as in the RA economy. Then they check that \bewley and \arrow, agents face the same set of Euler equations and complementarity slackness conditions
\end{frame}
%%%%%%%%%%%%%%%%%%%%%%%%%%%%%%%%%
\begin{frame}{equivalence results}

\end{frame}


